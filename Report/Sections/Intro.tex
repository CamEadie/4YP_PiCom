\documentclass[../main.tex]{subfiles}
	% INTRODUCTION
\begin{document}

This is about the project in general, what it does/did, and then about the report, how its structured and what it explains. Not sure how much of the "Conclusion-like" stuff should also be in here.

\todo[inline,color=blue!20]{Still need to write the Introduction Chapter}

%%%%%%%%%%%%%%%%%%%%%%%%%%%%%%%%%%%%%%%%%%%%%%%%%%%%%%%%%%%%%%%%%%%%%%%%%%%%%%%%%%%%%%%%%%%%%%%%%%%%%%%%%%%%%%%%%%%%%%%%%%%%%%%%%%

\section{Motivation}

Why is there a need for this, discuss the excessive cost of extant Software Defined Radio test beds, the low-cost alternative that is the Raspberry Pi. Use (PARAPHRASE) parts of Justin's description of the project as that sums it up nicely.

The section should do as you have organised: provide motivation for the project, define the project, give the background.  Remember your interrogatives: why, what, how.  This section relates mostly to why and what.  The 'how' is the subject of much of the rest of the report.

\begin{quotation}
	Modern digital communication systems are built upon a solid foundation of modulation and coding theory.
	Over the years, researchers have successfully developed numerous schemes using pen and paper along with computer models.
	Any such scheme ultimately must be tested on a suitable hardware/software platform to prove their usefulness in practice.
	Standard ‘software-defined radio’ test beds can cost thousands of pounds.
	Although these test beds provide users with advanced development tools, much of their functionality is superfluous to requirement.\\
	
	A Raspberry Pi is a simple, affordable ARM-based computer module that is capable of interfacing with external peripheral devices through a bank of IO ports.
	It is also programmable (using Python), and as such has found many uses by hobbyists and electronics/computer engineers in recent years.
	The purpose of this project is to develop a basic digital communication test bed using two Raspberry Pi modules (one transmitter and one receiver).
	The test bed will be affordable and the interested student will need to work to a budget to ensure a successful outcome.
	The project will require a considerable amount of Python programming as well as knowledge of, and a keen interest in, digital communication theory and techniques.
\end{quotation}


%%%%%%%%%%%%%%%%%%%%%%%%%%%%%%%%%%%%%%%%%%%%%%%%%%%%%%%%%%%%%%%%%%%%%%%%%%%%%%%%%%%%%%%%%%%%%%%%%%%%%%%%%%%%%%%%%%%%%%%%%%%%%%%%%%

\section{Background - Literature Review} \label{sec_Lit Review}

Explanation of the existing literature, what exists on the market and has this kind of project been attempted before to some degree by others.

Good.  I expect to see plenty of citations to related work here.  This will include not only Pi-based systems such as the paper I sent to you a while ago, but also NI kit, WARP demos, etc.

You will also need to lay out the basic communication-related tasks that you will focus on.  I.e., explain the need to develop this test bed for investigating modulation and coding schemes.  The degree of detail you give here is a matter of preference and flow.


%%%%%%%%%%%%%%%%%%%%%%%%%%%%%%%%%%%%%%%%%%%%%%%%%%%%%%%%%%%%%%%%%%%%%%%%%%%%%%%%%%%%%%%%%%%%%%%%%%%%%%%%%%%%%%%%%%%%%%%%%%%%%%%%%%

\section{Contributions}

My contributions to the existing literature, what difference I have made (in the third person...) - Write once I have finished writing up tests and results.

\end{document}  