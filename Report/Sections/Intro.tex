\documentclass[../main.tex]{subfiles}
	% INTRODUCTION
\begin{document}

The idea of a software radio is attributed to Dr. Joseph Mitola III in the early 1990's, referring to radios which could be reconfigured by changing the software, allowing for changes to communications protocols without changing the underlying hardware \cite{pap_Mitola}.
The ideal software radio would consist of a computer connected to a Digital-Analogue Converter (DAC) to an antenna as a transmitter, and an antenna connected to an Analogue-Digital Converter (ADC) to a computer as a receiver.
An ideal software radio is not realisable with the current converter technology, but with a few dedicated hardware components implementing the radio frequency (RF) front end, it is possible to develop a radio communication system which can implement a variety of alternative modulation schemes with little to no modification to the hardware simply by changing how the software operates.
This is the idea of a 'software-defined radio'.\\

'Software-defined radio' (SDR) test beds have become more popular as the technology becomes more accessible and cost-efficient.
These are test beds used to test the usefulness and effectiveness of various coding and modulation schemes for wireless communication.
Digital communication systems in the modern era are constantly evolving as researchers formulate new more efficient methods of transmitting data, approaching channel capacity.
These schemes are developed theoretically, but they eventually need to be tested in order to prove their utility, being able to physically test this with software instead of custom hardware is a significant advantage.
There are a number of available ‘software-defined radio’ test beds, however most of these can be very expensive.
Although these test beds supply advanced tool kits, the base functionality is really all that is required to successfully test new communication schemes.\\


%%%%%%%%%%%%%%%%%%%%%%%%%%%%%%%%%%%%%%%%%%%%%%%%%%%%%%%%%%%%%%%%%%%%%%%%%%%%%%%%%%%%%%%%%%%%%%%%%%%%%%%%%%%%%%%%%%%%%%%%%%%%%%%%%%

\section{Motivation}

The Raspberry Pi is a low-cost computer based on an ARM processor running a Linux distribution.
It has a number of software-programmable General Purpose Input/Output (GPIO) pins which can be used to interface with external devices.
This combination of computational power and versatility has  caught the attention of a wide range of engineers and hobbyists \cite{web_AboutPi}.
The Raspberry Pi provides an easily accessible way of developing an alternative SDR test bed which doesn't require thousands of pounds to purchase (see Section \ref{sec_Lit Review}).
This would make such test beds significantly more available, such that one might be built by anyone interested in investigating the physical realisation of their ideas.\\

Modern test beds often consist of radio transmitters and receivers wired together with adjustable attenuation between them.
This allows for the testing of these radio systems in close proximity, while simulating different distances between them \cite{pap_MilitaryRadioTB}.
The aim of this project is to develop a basic wired digital communications test bed between two Raspberry Pis, with one as the transmitter and the other as the receiver.
This will serve as a proof of concept for the development of low-budget software-defined radio test beds, and it is developed with costing in mind.
The next stage is then to attempt to characterise the performance of the test bed.
This consists of characterising the Raspberry Pi and other components used, as well as the test bed as pertains to its ability to communicate.
This includes testing its reliability for different modulation and coding schemes, and identifying key characteristics such as noise and bandwidth limitations.\\

%%%%%%%%%%%%%%%%%%%%%%%%%%%%%%%%%%%%%%%%%%%%%%%%%%%%%%%%%%%%%%%%%%%%%%%%%%%%%%%%%%%%%%%%%%%%%%%%%%%%%%%%%%%%%%%%%%%%%%%%%%%%%%%%%%

\section{Background - Literature Review} \label{sec_Lit Review}

There are a number of SDR test beds which have been developed in recent years.
A large number of them make use of one or more Ettus Universal Software Radio Peripherals (USRP), including the National Instruments Communications Kit which contains two such devices and is often used as an educational tool.
These USRPs are powerful tools, however the software defined radio peripherals range in price from \textit{£880} to \textit{£3,510}.
The reconfigurable SDR peripherals for rapid prototyping are even more expensive with prices starting around \textit{£5,000} \ref{web_NationalInstruments}.
An available free open source software which provides a communications libraries and tools and is used by a number of projects is GNU Radio. \cite{web_GNURadio:http://garethhayes.net/gnu-radio-rtl_sdr-raspberry-pi/}.
While this is a great option for a number of test beds, it does require a lot of processing power and space, which earlier Models of the Raspberry Pi couldn't handle but the Model 3+ used here is just about capable of managing.
However this would significantly impact the ability of the Raspberry Pi to also transmit and receive signals at as close to real-time performance rates as possible at the same time, unlike in USRP-based test beds where the Peripheral handles the load of transmitting/receiving and the computer separately manages the data conversion load of GNU Radio.
There have also been issues compiling GNU Radio on Raspberry Pis but this can be fixed \cite{web_GNUonRPi:http://garethhayes.net/gnu-radio-rtl_sdr-raspberry-pi/}.
Because of the issues with and the significant code base which would take time to master, the decision to develop independent software was made.
GNU Radio could possibly be included in an alternative iteration of this project's test bed.\\

In 2008 a group at the University of Notre Dame developed a portable software radio using only commercial off-the-shelf components, an Ettus USRP and GNU Radio \cite{web_PortableSR}.
It weighed 7 pounds and was the first portable software radio of its kind to their knowledge.
A single device made up of these components cost \textit{\$3,700} (about \textit{£1,900} at the time) and constituted a large development for the flexibility of radio systems.
Another group in 2014 using a USRP and GNU Radio developed the first SDR test bed for the testing of a RF subsampling receiver \cite{web_SDRTB_SubSamplingReceiver}.
This receiver sampled a \SI{4}{\giga\hertz} signal at \SI{100}{MS\per\second} with almost zero bit error and shows that radio systems are approaching the ideal software defined radio.
Another similar USRP-GNU Radio project in 2017 described a distributed software defined radio test bed for localisation and tracking of an emitter.\\

There are also projects which use a large number of these USRPs to simulate multiple interacting nodes in a radio communications network.
One of these is the DIWINE project, investigating wireless communication through a dense relay/node cluster.
In their final White Paper they described the use of six Ettus USRPs in a testing setup, two source, two relay and two destination nodes in a Smart Meter Network \cite{pap_DIWINEpaper4}.
Another similar USRP-GNU Radio project described a distributed software defined radio test bed for localisation and tracking of an emitter in real time \cite{pap_SDRTB_Localisation}.
This used a complex combination of GPS synchronisation and Kalman filtering to track the target node (USRP) based on readings from a number of distributed USRP-based sensors. 
An alternative to the Ettus USRP is the WARP (Wireless Open-Access Research Platform) Board developed at Rice University.
It includes two programmable RF interfaces and a number of peripherals.
A demonstration using WARP Boards for a software-defined visible light communications system was presented by a two researchers from the Rice University and one from The University of Edinburgh \cite{pap_WARP_light}.
This system was used to demonstrate optical Orthogonal Frequency Division Multiplexing (OFDM) using a Warp Board connected to an LED as the transmitter and .
The WARP v3 kit, which contains the WARP v3 Board, power supply and SD card, costs \textit{\$4,900} for academic customers or \textit{\$6,900} for other customers (about \textit{£3,600} and \textit{£5,000} respectively).\\

The problem with these solutions is that they are all prohibitively expensive, meaning that not as many researchers have access to SDR test beds as they would like.
As a result, particularly in the last two years, a number of groups have looked to the Raspberry Pi as a low-cost alternative to develop SDR test beds.
The first examples of this are not designed using the Raspberry Pi as the main component of the test bed but as a low cost computer to interface with still-used USRPs.


spectrum \cite{pap_PiSDRTB_Spectrum}

WiPi:
\cite{pap_WiPi} 

education \cite{pap_PiSimulinkEducation}

\


\begin{quotation}
	You will also need to lay out the basic communication-related tasks that you will focus on.  I.e., explain the need to develop this test bed for investigating modulation and coding schemes.  The degree of detail you give here is a matter of preference and flow.
\end{quotation}



%%%%%%%%%%%%%%%%%%%%%%%%%%%%%%%%%%%%%%%%%%%%%%%%%%%%%%%%%%%%%%%%%%%%%%%%%%%%%%%%%%%%%%%%%%%%%%%%%%%%%%%%%%%%%%%%%%%%%%%%%%%%%%%%%%

\section{Contributions}

The body of this report is structured into three main chapters.
Chapter \ref{sec_RPi}, "The Raspberry Pi and the Test Bed" describes how the Raspberry Pi is set up for its operation in the test bed and what comprises the physical architecture, as well as investigating the code which was developed to test various modulation schemes.
Chapter \ref{sec_Electro Testing}, "Electronic Testing" is aimed at characterising the Raspberry Pi electrically and computationally to understand its capabilities and limitations.
It also characterises the physical external components used.
Chapter \ref{sec_Comms Testing}, "Communications Testing" runs through tests for different modulation and coding schemes at different frequencies in order to characterise the test bed in a communications context, as well as discover its limits.\\

My contributions to the existing literature, what difference I have made (in the third person...) - Write once I have finished writing up tests and results.
\todo[inline,color=blue!20]{Still need to write the Contributions Chapter (last)}

\end{document}  